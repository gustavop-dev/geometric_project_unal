\section{Resultados}

La implementación del sistema de segmentación y modelado 3D del húmero produjo resultados notables, especialmente después de las mejoras realizadas para mantener la coherencia anatómica entre cortes.

\subsection{Segmentación mejorada}

\subsubsection{Evaluación cualitativa}

La evaluación cualitativa de los resultados mostró una mejora significativa en la calidad de la segmentación:

\begin{itemize}
    \item \textbf{Continuidad anatómica:} Los contornos detectados en cortes consecutivos mostraron una transición suave y anatómicamente coherente, eliminando las discontinuidades presentes en versiones anteriores.
    
    \item \textbf{Forma anatómica:} Los splines generados siguieron fielmente la forma semicircular del húmero, incluso en las regiones problemáticas como las imágenes 6 y 7, donde anteriormente aparecían como líneas rectas.
    
    \item \textbf{Robustez:} El sistema demostró ser robusto ante variaciones en la calidad de la imagen y en las características anatómicas del húmero a lo largo de toda su longitud.
\end{itemize}

\subsubsection{Casos problemáticos}

En particular, la mejora en los casos problemáticos (imágenes 6 y 7) fue notable:

\begin{itemize}
    \item \textbf{Antes:} Los splines aparecían como líneas rectas, sin seguir el contorno semicircular del húmero.
    \item \textbf{Después:} Los splines siguen correctamente la forma anatómica del húmero, manteniendo la coherencia con los cortes adyacentes.
\end{itemize}

Esta mejora se logró mediante:

\begin{enumerate}
    \item El uso del contorno del corte anterior como referencia.
    \item El tratamiento especial de estos cortes específicos con parámetros ajustados.
    \item El aumento del factor de suavizado en los B-splines.
    \item La prevención del colapso del algoritmo snake mediante el aumento de la tensión y rigidez.
\end{enumerate}

\subsection{Modelo 3D}

El modelo 3D generado a partir de los contornos mejorados presenta una representación coherente y anatómicamente precisa del húmero:

\begin{itemize}
    \item \textbf{Superficie suave:} La superficie del modelo 3D muestra transiciones suaves entre los diferentes cortes.
    \item \textbf{Consistencia anatómica:} El modelo refleja la forma tubular del cuerpo del húmero y sus variaciones a lo largo de toda su longitud.
    \item \textbf{Visualización completa:} La posibilidad de rotar el modelo permite observar todas las características anatómicas desde diferentes ángulos.
\end{itemize}

\subsection{Ventajas del enfoque implementado}

Las principales ventajas del enfoque implementado son:

\begin{enumerate}
    \item \textbf{Robustez:} El sistema funciona correctamente incluso en presencia de variaciones anatómicas y artefactos en las imágenes.
    
    \item \textbf{Continuidad:} La transferencia de información entre cortes consecutivos asegura la coherencia anatómica en todo el modelo.
    
    \item \textbf{Adaptabilidad:} El sistema se adapta a diferentes regiones del húmero, aplicando estrategias específicas según las características anatómicas.
    
    \item \textbf{Suavidad:} El uso de B-splines con parámetros optimizados proporciona contornos suaves y precisos.
    
    \item \textbf{Visualización intuitiva:} El modelo 3D y sus visualizaciones permiten una comprensión clara de la anatomía del húmero.
\end{enumerate}

\subsection{Limitaciones y trabajo futuro}

A pesar de los buenos resultados, se identificaron algunas limitaciones y oportunidades de mejora:

\begin{itemize}
    \item \textbf{Dependencia de parámetros manuales:} Algunos parámetros, como el factor de suavizado de los splines, requieren ajuste manual para diferentes conjuntos de imágenes.
    
    \item \textbf{Tratamiento especial para casos específicos:} La necesidad de tratamiento especial para ciertos cortes indica que podría ser beneficioso desarrollar un enfoque más generalizado.
    
    \item \textbf{Validación cuantitativa:} Sería valioso implementar métricas de evaluación cuantitativa para comparar la segmentación con segmentaciones manuales realizadas por expertos.
\end{itemize}

Para trabajo futuro, se propone:

\begin{enumerate}
    \item Desarrollar un sistema de aprendizaje para ajustar automáticamente los parámetros según las características de cada imagen.
    
    \item Extender el enfoque a otras estructuras anatómicas, como otros huesos largos o estructuras más complejas.
    
    \item Implementar métodos de validación cuantitativa para evaluar precisión y exactitud.
    
    \item Explorar tecnologías de aprendizaje profundo para la segmentación, que podrían complementar el enfoque basado en procesamiento de imágenes actual.
\end{enumerate} 
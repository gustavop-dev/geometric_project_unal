\documentclass[12pt,a4paper]{article}
\usepackage[spanish]{babel}
\usepackage[utf8]{inputenc}
\usepackage{amsmath,amsfonts,amssymb}
\usepackage{graphicx}
\usepackage{xcolor}
\usepackage{geometry}
\usepackage{hyperref}
\usepackage{listings}
\usepackage{float}
\usepackage{subcaption}
\usepackage{algorithm}
\usepackage{algpseudocode}

\geometry{margin=2.5cm}

\hypersetup{
    colorlinks=true,
    linkcolor=blue,
    filecolor=magenta,
    urlcolor=blue,
}

\lstset{
    basicstyle=\small\ttfamily,
    commentstyle=\color{gray},
    keywordstyle=\color{blue},
    stringstyle=\color{red},
    breaklines=true,
    showstringspaces=false,
    numbers=left,
    numberstyle=\tiny\color{gray},
    frame=single,
}

\title{\textbf{Segmentación y Modelado 3D del Húmero a partir de Imágenes de Tomografía Computarizada}}
\author{Estudiante de Geometría Computacional}
\date{\today}

\begin{document}

\maketitle

\begin{abstract}
Este informe documenta el desarrollo, implementación y mejora de un sistema de segmentación y modelado 3D del húmero utilizando imágenes de tomografía computarizada. El proyecto aborda diversos desafíos en la detección de contornos, corrección de errores de segmentación, y visualización tridimensional coherente. Se detallan las técnicas de procesamiento de imágenes médicas, algoritmos de detección de contornos, y métodos de interpolación mediante splines utilizados para reconstruir la estructura anatómica del húmero a partir de cortes axiales.
\end{abstract}

\tableofcontents

\newpage

\section{Introducción}

La reconstrucción tridimensional de estructuras anatómicas a partir de imágenes médicas representa un desafío importante en el campo de la medicina y la computación gráfica. Este proyecto se centra en la segmentación del húmero (el hueso más largo del brazo) a partir de imágenes de tomografía computarizada (TC) axiales y su posterior modelado 3D.

\subsection{Contexto del problema}

Las imágenes de tomografía computarizada proporcionan cortes axiales del cuerpo humano con alta resolución y contraste, lo que permite visualizar estructuras óseas como el húmero con claridad. Sin embargo, la segmentación automática de estas estructuras presenta varios retos:

\begin{itemize}
    \item \textbf{Variabilidad anatómica:} La forma del húmero puede variar entre individuos.
    \item \textbf{Discontinuidades entre cortes:} Cada corte axial representa una sección diferente del húmero, y mantener la coherencia anatómica entre cortes consecutivos resulta complejo.
    \item \textbf{Regiones problemáticas:} Ciertas áreas del húmero, como las extremidades proximal y distal, presentan cambios rápidos en su morfología que dificultan la detección automática.
    \item \textbf{Presencia de tejidos adyacentes:} Otros tejidos cercanos pueden dificultar la detección precisa del contorno del húmero.
\end{itemize}

\subsection{Objetivos}

Los objetivos principales de este proyecto son:

\begin{enumerate}
    \item Desarrollar un algoritmo robusto para la detección de contornos del húmero en cortes axiales de TC.
    \item Implementar técnicas de ajuste mediante splines para representar fielmente la forma del húmero en cada corte.
    \item Asegurar la coherencia anatómica entre cortes consecutivos para una reconstrucción 3D precisa.
    \item Generar un modelo tridimensional del húmero que permita su visualización desde diferentes ángulos.
\end{enumerate}

\subsection{Desafío específico}

Durante el desarrollo del proyecto, se identificó un problema específico en la segmentación: los splines generados entre cortes axiales consecutivos no mantenían coherencia anatómica, especialmente en algunas imágenes donde el spline aparecía como una línea en lugar de seguir la forma semicircular del húmero. Este informe documenta las soluciones implementadas para resolver este y otros desafíos encontrados durante el proceso de segmentación y modelado. 
\section{Metodología}

La metodología desarrollada para la segmentación y modelado 3D del húmero consta de varias etapas secuenciales que trabajan de forma integrada para conseguir resultados anatómicamente precisos.

\subsection{Flujo de trabajo general}

El procesamiento de las imágenes sigue el siguiente flujo de trabajo:

\begin{enumerate}
    \item \textbf{Lectura de imágenes DICOM:} Carga de los cortes axiales en formato DICOM, extrayendo tanto los datos de los píxeles como los metadatos asociados (espaciado entre cortes, resolución de píxel, etc.).
    \item \textbf{Preprocesamiento:} Normalización, filtrado y realce de contraste para mejorar la visibilidad del húmero.
    \item \textbf{Detección inicial:} Identificación de la región donde se encuentra el húmero utilizando la transformada circular de Hough.
    \item \textbf{Segmentación avanzada:} Determinación precisa del contorno del húmero mediante técnicas de contorno activo (snakes).
    \item \textbf{Ajuste mediante splines:} Aplicación de splines para obtener un contorno suave y anatómicamente correcto.
    \item \textbf{Detección y corrección de valores atípicos:} Identificación y corrección de irregularidades en los contornos.
    \item \textbf{Integración de continuidad entre cortes:} Uso del contorno del corte anterior como referencia para mejorar la coherencia.
    \item \textbf{Generación del modelo 3D:} Integración de todos los contornos para construir un modelo tridimensional completo.
    \item \textbf{Visualización:} Representación del modelo 3D con técnicas de renderizado.
\end{enumerate}

\subsection{Técnicas de procesamiento de imagen}

\subsubsection{Detección de contornos}

Para la detección inicial del húmero, se utiliza la transformada circular de Hough, que permite identificar estructuras circulares en la imagen. Este enfoque es adecuado ya que el húmero aparece aproximadamente circular en los cortes axiales.

El algoritmo implementado sigue estos pasos:
\begin{enumerate}
    \item Normalización de la imagen y inversión (el húmero aparece como región oscura).
    \item Aplicación de filtro gaussiano para reducción de ruido.
    \item Detección de bordes mediante el algoritmo de Canny.
    \item Aplicación de la transformada de Hough para círculos, con radios adaptados al tamaño esperado del húmero.
    \item Evaluación de los círculos detectados basándose en su posición y características de intensidad.
\end{enumerate}

\subsubsection{Contornos activos (Snakes)}

Una vez identificada la región aproximada del húmero, se utiliza el algoritmo de contornos activos para refinar la segmentación:

\begin{enumerate}
    \item Se inicializa el contorno activo con el círculo detectado previamente.
    \item Se configura el algoritmo con parámetros de tensión y rigidez adaptados.
    \item Se itera el proceso de deformación del contorno hasta que converja a los bordes del húmero.
    \item Se aplican restricciones para evitar colapsos o expansiones excesivas del contorno.
\end{enumerate}

\subsection{Continuidad entre cortes}

Para asegurar la coherencia anatómica entre cortes consecutivos, se implementaron las siguientes estrategias:

\begin{enumerate}
    \item \textbf{Transferencia de contorno:} El contorno detectado en un corte se utiliza como inicialización para el siguiente corte.
    \item \textbf{Tratamiento especial para regiones problemáticas:} Se identificaron las imágenes que presentaban dificultades (específicamente las imágenes 6 y 7) y se les aplicó un procesamiento adaptado.
    \item \textbf{Corrección de posición:} Se implementó un algoritmo para ajustar la posición del contorno basándose en la coherencia anatómica esperada.
    \item \textbf{Mecanismos de respaldo:} Cuando la detección falla en un corte, se utiliza una versión modificada del contorno anterior.
\end{enumerate}

\subsection{Ajuste mediante splines}

Para obtener contornos suaves y anatómicamente precisos, se implementó un sistema de ajuste mediante B-splines:

\begin{enumerate}
    \item Se procesan los puntos del contorno para eliminar valores atípicos.
    \item Se parametriza el contorno según la longitud de arco.
    \item Se aplica un B-spline cerrado con grado 3 y un factor de suavizado adaptativo.
    \item Se evalúa el spline en puntos equidistantes para obtener un contorno uniforme.
\end{enumerate}

\subsection{Modelado 3D}

Para la generación del modelo 3D, se siguen estos pasos:

\begin{enumerate}
    \item Extracción de los contornos de splines de las imágenes resultantes.
    \item Normalización de cada contorno a un número uniforme de puntos.
    \item Asignación de coordenadas Z basadas en el espaciado entre cortes de las imágenes DICOM.
    \item Conexión de contornos consecutivos para formar una superficie tridimensional.
    \item Generación de visualizaciones estáticas y animadas (GIF) con rotación del modelo 3D.
\end{enumerate} 
\section{Implementación}

La implementación del sistema se organizó de manera modular para facilitar el mantenimiento y la extensibilidad. A continuación, se detallan los principales componentes y su implementación.

\subsection{Estructura del código}

El código se organizó en una estructura jerárquica de módulos y submódulos:

\begin{itemize}
    \item \textbf{io:} Módulos para la lectura y escritura de archivos DICOM y manejo de metadatos.
    \item \textbf{preprocessing:} Funciones para el preprocesamiento de imágenes (normalización, filtrado).
    \item \textbf{contour:} Algoritmos para la detección de contornos y procesamiento geométrico.
    \item \textbf{spline:} Implementación de ajuste mediante B-splines.
    \item \textbf{visualization:} Herramientas para visualización de resultados y generación de modelos 3D.
\end{itemize}

\subsection{Scripts principales}

Se implementaron dos scripts principales:

\begin{itemize}
    \item \textbf{run\_detection.py:} Ejecuta el pipeline completo de detección del húmero en todas las imágenes DICOM.
    \item \textbf{run\_visualization.py:} Genera el modelo 3D a partir de los resultados de la detección.
\end{itemize}

\subsection{Pipeline de procesamiento}

El pipeline implementado en el módulo principal sigue esta secuencia:

\begin{lstlisting}[language=Python, caption=Pipeline principal de procesamiento]
def run_pipeline(dicom_directory: str, output_directory: str, show_images: bool = False) -> None:
    """
    Run the complete humerus detection and modeling pipeline.
    
    Args:
        dicom_directory: Directory containing the DICOM files
        output_directory: Directory where to save the results
        show_images: Whether to display the images during processing
    """
    # Create output directory if it doesn't exist
    os.makedirs(output_directory, exist_ok=True)
    
    # Process all DICOM files in the directory
    files = sorted([f for f in os.listdir(dicom_directory) if f.endswith('.dcm')])
    previous_contour = None
    
    for i, file in enumerate(files):
        # Step 1: Read DICOM file
        dataset, pixel_array = dicom_utils.read_dicom(full_path)
            
        # Step 2: Detect the humerus contour
        contour = detection.detect_advanced_humerus_contour(
            pixel_array, 
            previous_contour=previous_contour, 
            file_name=base_name,
            slice_height=dicom_utils.get_slice_height(dataset)
        )
            
        # Step 3: Apply B-spline
        smoothed_contour = outliers.detect_and_correct_outliers(contour, is_spline=False)
        x_spline, y_spline = fitting.apply_bspline(smoothed_contour, degree, smoothing)
        
        # Step 4: Visualize and save results
        fig = plotting.visualize_results(pixel_array, contour, x_spline, y_spline, title)
        plotting.save_results(fig, output_directory, base_name, "advanced")
        
        # Update previous contour for next iteration
        previous_contour = contour
\end{lstlisting}

\subsection{Detección avanzada del húmero}

El algoritmo de detección avanzada incluye múltiples estrategias para asegurar la robustez:

\begin{lstlisting}[language=Python, caption=Detección avanzada del húmero]
def detect_advanced_humerus_contour(
    pixel_array: np.ndarray, 
    previous_contour: Optional[np.ndarray] = None, 
    file_name: Optional[str] = None,
    slice_height: Optional[float] = None
) -> np.ndarray:
    # Tratamiento especial para casos problemáticos
    if file_name in ["I6", "I7"]:
        # Aplicar técnica específica para estas imágenes
        # ... (código específico)
    
    # Intentar usar el contorno previo como referencia
    if previous_contour is not None:
        # Ajustar ligeramente el contorno anterior para adaptarlo al corte actual
        init_contour = adjust_from_previous_contour(previous_contour, pixel_array)
        # ... (código de refinamiento)
    else:
        # Si no hay contorno previo, usar detección de círculo
        center_x, center_y, radius = detect_humerus_circle(pixel_array)
        if center_x is not None:
            init_contour = generate_circular_contour(center_x, center_y, radius)
    
    # Aplicar contorno activo (snake) para refinar
    contour = apply_snake_algorithm(init_contour, pixel_array, 
                                     alpha=0.015, beta=10.0,  # Tensión y rigidez
                                     gamma=0.001,            # Paso temporal
                                     iterations=100)
\end{lstlisting}

\subsection{Ajuste mediante B-splines}

Para el ajuste de B-splines, se implementó una función que maneja el cálculo de la parametrización y el ajuste del spline:

\begin{lstlisting}[language=Python, caption=Implementación de B-splines]
def apply_bspline(contour: np.ndarray, degree: int = 3, smoothing: float = 10.0) -> Tuple[np.ndarray, np.ndarray]:
    """
    Apply B-spline fitting to a contour.
    
    Args:
        contour: Input contour points as numpy array of shape (n_points, 2)
        degree: Degree of the B-spline
        smoothing: Smoothing factor (0 = interpolation)
        
    Returns:
        Tuple containing arrays of x and y coordinates of the spline
    """
    try:
        # Extract x and y coordinates
        y, x = contour[:, 0], contour[:, 1]
        
        # Arc length parametrization
        t = np.zeros(len(x))
        t[1:] = np.cumsum(np.sqrt(np.diff(x)**2 + np.diff(y)**2))
        t = t / t[-1]
        
        # Periodic B-spline fitting
        tck, u = splprep([x, y], u=t, s=smoothing, k=degree, per=1)
        
        # Evaluate the spline at uniform parameter values
        u_new = np.linspace(0, 1, 200)
        x_spline, y_spline = splev(u_new, tck)
        
        return x_spline, y_spline
    except Exception as e:
        print(f"Error applying B-spline: {e}")
        return None, None
\end{lstlisting}

\subsection{Generación del modelo 3D}

La generación del modelo 3D se realiza extrayendo los contornos de las imágenes procesadas y combinándolos en un espacio tridimensional:

\begin{lstlisting}[language=Python, caption=Generación del modelo 3D]
def generate_3d_model(contours, z_spacing, pixel_spacing, output_path):
    # Crear figura 3D
    fig = plt.figure(figsize=(12, 10))
    ax = fig.add_subplot(111, projection='3d')
    
    # Para cada contorno, asignar una coordenada Z y dibujar
    for i, contour in enumerate(valid_contours):
        # Z coordinate based on DICOM spacing
        z = i * z_spacing
        
        # Scale coordinates according to pixel spacing
        x = contour[:, 1] * pixel_spacing[0]
        y = contour[:, 0] * pixel_spacing[1]
        
        # Resample with B-splines for uniformity
        t = np.zeros(len(x))
        t[1:] = np.cumsum(np.sqrt(np.diff(x)**2 + np.diff(y)**2))
        t = t / t[-1]
        
        # Draw contour
        tck, u = splprep([x, y], u=t, s=0, per=1)
        u_new = np.linspace(0, 1, n_points)
        x_new, y_new = splev(u_new, tck)
        ax.plot(x_new, y_new, [z]*len(x_new), '-', color=colors[i])
        
        # Connect with previous slice
        if i > 0:
            # ... (código para conectar con el corte anterior)
\end{lstlisting}

\subsection{Mejoras implementadas}

Para resolver los problemas específicos de coherencia anatómica, se realizaron las siguientes mejoras:

\begin{enumerate}
    \item \textbf{Transferencia de contorno mejorada:} Se implementó un algoritmo más avanzado para ajustar el contorno previo al corte actual, teniendo en cuenta posibles cambios en la forma y posición.
    
    \item \textbf{Aumento de rigidez en snakes:} Se incrementaron los parámetros de tensión y rigidez del algoritmo de contornos activos para prevenir colapsos y mantener una forma más estable.
    
    \item \textbf{Detección específica por altura del corte:} Se ajustaron los parámetros de detección según la altura del corte, aplicando diferentes estrategias para diferentes regiones del húmero.
    
    \item \textbf{Mejor parametrización de splines:} Se mejoró la parametrización por longitud de arco para obtener distribuciones más uniformes de los puntos del spline.
    
    \item \textbf{Mayor factor de suavizado:} Se aumentó el factor de suavizado en los B-splines para obtener contornos más suaves y anatómicamente verosímiles.
\end{enumerate} 
\section{Resultados}

La implementación del sistema de segmentación y modelado 3D del húmero produjo resultados notables, especialmente después de las mejoras realizadas para mantener la coherencia anatómica entre cortes.

\subsection{Segmentación mejorada}

\subsubsection{Evaluación cualitativa}

La evaluación cualitativa de los resultados mostró una mejora significativa en la calidad de la segmentación:

\begin{itemize}
    \item \textbf{Continuidad anatómica:} Los contornos detectados en cortes consecutivos mostraron una transición suave y anatómicamente coherente, eliminando las discontinuidades presentes en versiones anteriores.
    
    \item \textbf{Forma anatómica:} Los splines generados siguieron fielmente la forma semicircular del húmero, incluso en las regiones problemáticas como las imágenes 6 y 7, donde anteriormente aparecían como líneas rectas.
    
    \item \textbf{Robustez:} El sistema demostró ser robusto ante variaciones en la calidad de la imagen y en las características anatómicas del húmero a lo largo de toda su longitud.
\end{itemize}

\subsubsection{Casos problemáticos}

En particular, la mejora en los casos problemáticos (imágenes 6 y 7) fue notable:

\begin{itemize}
    \item \textbf{Antes:} Los splines aparecían como líneas rectas, sin seguir el contorno semicircular del húmero.
    \item \textbf{Después:} Los splines siguen correctamente la forma anatómica del húmero, manteniendo la coherencia con los cortes adyacentes.
\end{itemize}

Esta mejora se logró mediante:

\begin{enumerate}
    \item El uso del contorno del corte anterior como referencia.
    \item El tratamiento especial de estos cortes específicos con parámetros ajustados.
    \item El aumento del factor de suavizado en los B-splines.
    \item La prevención del colapso del algoritmo snake mediante el aumento de la tensión y rigidez.
\end{enumerate}

\subsection{Modelo 3D}

El modelo 3D generado a partir de los contornos mejorados presenta una representación coherente y anatómicamente precisa del húmero:

\begin{itemize}
    \item \textbf{Superficie suave:} La superficie del modelo 3D muestra transiciones suaves entre los diferentes cortes.
    \item \textbf{Consistencia anatómica:} El modelo refleja la forma tubular del cuerpo del húmero y sus variaciones a lo largo de toda su longitud.
    \item \textbf{Visualización completa:} La posibilidad de rotar el modelo permite observar todas las características anatómicas desde diferentes ángulos.
\end{itemize}

\subsection{Ventajas del enfoque implementado}

Las principales ventajas del enfoque implementado son:

\begin{enumerate}
    \item \textbf{Robustez:} El sistema funciona correctamente incluso en presencia de variaciones anatómicas y artefactos en las imágenes.
    
    \item \textbf{Continuidad:} La transferencia de información entre cortes consecutivos asegura la coherencia anatómica en todo el modelo.
    
    \item \textbf{Adaptabilidad:} El sistema se adapta a diferentes regiones del húmero, aplicando estrategias específicas según las características anatómicas.
    
    \item \textbf{Suavidad:} El uso de B-splines con parámetros optimizados proporciona contornos suaves y precisos.
    
    \item \textbf{Visualización intuitiva:} El modelo 3D y sus visualizaciones permiten una comprensión clara de la anatomía del húmero.
\end{enumerate}

\subsection{Limitaciones y trabajo futuro}

A pesar de los buenos resultados, se identificaron algunas limitaciones y oportunidades de mejora:

\begin{itemize}
    \item \textbf{Dependencia de parámetros manuales:} Algunos parámetros, como el factor de suavizado de los splines, requieren ajuste manual para diferentes conjuntos de imágenes.
    
    \item \textbf{Tratamiento especial para casos específicos:} La necesidad de tratamiento especial para ciertos cortes indica que podría ser beneficioso desarrollar un enfoque más generalizado.
    
    \item \textbf{Validación cuantitativa:} Sería valioso implementar métricas de evaluación cuantitativa para comparar la segmentación con segmentaciones manuales realizadas por expertos.
\end{itemize}

Para trabajo futuro, se propone:

\begin{enumerate}
    \item Desarrollar un sistema de aprendizaje para ajustar automáticamente los parámetros según las características de cada imagen.
    
    \item Extender el enfoque a otras estructuras anatómicas, como otros huesos largos o estructuras más complejas.
    
    \item Implementar métodos de validación cuantitativa para evaluar precisión y exactitud.
    
    \item Explorar tecnologías de aprendizaje profundo para la segmentación, que podrían complementar el enfoque basado en procesamiento de imágenes actual.
\end{enumerate} 
\section{Conclusiones}

El desarrollo e implementación del sistema de segmentación y modelado 3D del húmero ha permitido obtener valiosas conclusiones sobre el procesamiento de imágenes médicas y la reconstrucción tridimensional de estructuras anatómicas.

\subsection{Principales logros}

Se han alcanzado los siguientes logros principales:

\begin{enumerate}
    \item \textbf{Segmentación precisa:} Se ha logrado una segmentación precisa del húmero en imágenes de tomografía computarizada, incluso en regiones anatómicamente complejas.
    
    \item \textbf{Coherencia anatómica:} Se ha conseguido mantener la coherencia anatómica entre cortes consecutivos, generando una representación tridimensional anatómicamente correcta.
    
    \item \textbf{Robustez del algoritmo:} El algoritmo desarrollado ha demostrado ser robusto ante variaciones en la calidad de la imagen y cambios en la morfología del húmero.
    
    \item \textbf{Visualización efectiva:} El modelo 3D generado permite una visualización efectiva y comprensible de la estructura anatómica del húmero.
\end{enumerate}

\subsection{Implicaciones técnicas}

Desde el punto de vista técnico, el proyecto ha demostrado:

\begin{enumerate}
    \item La efectividad de combinar técnicas de detección inicial (transformada de Hough) con métodos de refinamiento de contorno (snake) para segmentación médica.
    
    \item La importancia de la transferencia de información entre cortes consecutivos para mantener la coherencia anatómica.
    
    \item El valor de los B-splines como herramienta para obtener representaciones suaves y precisas de estructuras anatómicas.
    
    \item La necesidad de adaptar los algoritmos a las características específicas de diferentes regiones anatómicas.
\end{enumerate}

\subsection{Impacto potencial}

El sistema desarrollado tiene potencial para impactar en diversos campos:

\begin{itemize}
    \item \textbf{Planificación quirúrgica:} La reconstrucción 3D precisa del húmero puede ser útil para la planificación de intervenciones quirúrgicas.
    
    \item \textbf{Educación médica:} Los modelos generados pueden utilizarse como herramientas educativas para estudiantes de medicina y profesionales de la salud.
    
    \item \textbf{Investigación biomédica:} La metodología puede adaptarse para estudios de variabilidad anatómica o análisis de patologías específicas.
    
    \item \textbf{Desarrollo de prótesis:} Los modelos 3D precisos pueden contribuir al diseño de prótesis personalizadas.
\end{itemize}

\subsection{Lecciones aprendidas}

El desarrollo del proyecto ha proporcionado valiosas lecciones:

\begin{enumerate}
    \item \textbf{Importancia del enfoque iterativo:} La identificación de problemas específicos y su resolución mediante mejoras incrementales ha sido clave para el éxito del proyecto.
    
    \item \textbf{Valor del conocimiento anatómico:} La comprensión de la anatomía del húmero ha sido fundamental para evaluar los resultados y guiar las mejoras del algoritmo.
    
    \item \textbf{Necesidad de redundancia:} La implementación de mecanismos de respaldo para cuando la detección falla ha aumentado significativamente la robustez del sistema.
    
    \item \textbf{Equilibrio entre parametrización manual y automatización:} Si bien es deseable un alto grado de automatización, ciertos parámetros requieren ajuste manual para adaptarse a diferentes casos.
\end{enumerate}

\subsection{Consideraciones finales}

El proyecto demuestra que es posible obtener reconstrucciones 3D anatómicamente precisas a partir de imágenes de tomografía computarizada utilizando técnicas de procesamiento de imagen y modelado geométrico. La metodología desarrollada proporciona una base sólida que puede extenderse a otras estructuras anatómicas y aplicaciones médicas.

La combinación de algoritmos de detección de contornos, contornos activos, ajuste mediante splines y visualización 3D ha demostrado ser efectiva para la segmentación y modelado del húmero, superando desafíos como la coherencia anatómica entre cortes y la precisión en regiones complejas.

Este trabajo establece una metodología viable para la reconstrucción tridimensional a partir de imágenes médicas que podría integrarse en sistemas de apoyo a la práctica clínica, la educación médica y la investigación biomédica. 

\appendix
\section{Código fuente}
\section{Fragmentos de código relevantes}

A continuación se presentan fragmentos clave del código desarrollado para la segmentación y modelado 3D del húmero.

\subsection{Pipeline principal}

El siguiente fragmento muestra la estructura principal del pipeline de procesamiento:

\begin{lstlisting}[language=Python, caption=Pipeline de procesamiento en humerus\_detection/\_\_init\_\_.py]
def run_pipeline(dicom_directory: str, output_directory: str, show_images: bool = False) -> None:
    """
    Run the complete humerus detection and modeling pipeline.
    
    Args:
        dicom_directory: Directory containing the DICOM files
        output_directory: Directory where to save the results
        show_images: Whether to display the images during processing
    """
    print(f"Running humerus detection pipeline:")
    print(f"  - DICOM directory: {dicom_directory}")
    print(f"  - Output directory: {output_directory}")
    
    # Create output directory if it doesn't exist
    os.makedirs(output_directory, exist_ok=True)
    
    # Process all DICOM files in the directory
    files = sorted([f for f in os.listdir(dicom_directory) if f.endswith('.dcm')])
    
    print(f"Found {len(files)} DICOM files to process")
    previous_contour = None
    
    for i, file in enumerate(files):
        print(f"Processing {i+1}/{len(files)}: {file}")
        full_path = os.path.join(dicom_directory, file)
        
        # Step 1: Read DICOM file
        dataset, pixel_array = dicom_utils.read_dicom(full_path)
        if dataset is None or pixel_array is None:
            print(f"  Error reading {file}")
            continue
        
        base_name = os.path.splitext(os.path.basename(full_path))[0]
            
        # Step 2: Detect the humerus contour
        contour = detection.detect_advanced_humerus_contour(
            pixel_array, 
            previous_contour=previous_contour, 
            file_name=base_name,
            slice_height=dicom_utils.get_slice_height(dataset)
        )
        
        # Step 3: If no contour (humerus disappeared), continue to next file
        if len(contour) == 0:
            print(f"  No humerus detected in {base_name}")
            previous_contour = None
            
            # Save an empty image for visualization
            fig = plt.figure(figsize=(10, 8))
            plt.imshow(pixel_array, cmap='gray')
            plt.title(f"Axial slice - No humerus detected")
            plt.axis('off')
            plt.tight_layout()
            plotting.save_results(fig, output_directory, base_name, "end_humerus")
            plt.close(fig)
            continue
            
        # Step 4: Apply B-spline
        smoothed_contour = outliers.detect_and_correct_outliers(contour, is_spline=False)
        
        # Use more smoothing for a more anatomical result
        smoothing = 10.0 * 2.5
        degree = 3
        x_spline, y_spline = fitting.apply_bspline(smoothed_contour, degree, smoothing)
        
        if x_spline is None or y_spline is None:
            print(f"  Error applying B-spline to {base_name}")
            continue
        
        # Step 5: Visualize and save results
        title = f"Axial slice - B-spline (degree {degree}, smoothing {smoothing})"
        fig = plotting.visualize_results(
            pixel_array, contour, x_spline, y_spline, title
        )
        
        plotting.save_results(fig, output_directory, base_name, "advanced")
        
        if not show_images:
            plt.close(fig)
        else:
            plt.show()
        
        # Update previous contour for next iteration
        previous_contour = contour
\end{lstlisting}

\subsection{Detección del húmero}

El algoritmo de detección avanzada del húmero incluye mecanismos para mantener coherencia anatómica:

\begin{lstlisting}[language=Python, caption=Fragmento de detect\_advanced\_humerus\_contour en contour/detection.py]
def detect_advanced_humerus_contour(
    pixel_array: np.ndarray, 
    previous_contour: Optional[np.ndarray] = None, 
    file_name: Optional[str] = None,
    slice_height: Optional[float] = None
) -> np.ndarray:
    """
    Detects the humerus contour (dark region) in a DICOM image.
    1. Detects the circular zone (humerus) with Hough.
    2. Searches for the real contour only within that zone.
    3. Adjusts the contour with snake.
    4. If it fails, uses the circle as a fallback.
    
    Includes improvements to maintain continuity between images.
    
    Args:
        pixel_array: Pixel matrix of the DICOM image
        previous_contour: Contour from the previous slice for continuity
        file_name: Name of the file being processed (used for special handling of terminal slices)
        slice_height: Height of the slice in mm
        
    Returns:
        Detected contour as a numpy array of shape (n_points, 2)
    """
    # Check if we're in the final slices where the humerus disappears
    final_slices = ["I18", "I19", "I20"]
    
    # If we're in a final slice, force the termination of the humerus
    if file_name in final_slices and previous_contour is not None:
        # Generate a terminal contour that gets smaller with each slice
        slice_idx = final_slices.index(file_name)
        reduction_factor = 0.5 - (slice_idx * 0.2)  # From 0.5 to 0.1
        
        return generate_terminal_contour(previous_contour, reduction_factor)
    
    # Special handling for problematic images
    if file_name in ["I6", "I7"]:
        # Use higher threshold and adjusted parameters
        norm_image = image_processing.normalize_image(pixel_array)
        
        # Apply threshold with higher value for these specific images
        threshold = 0.3 if file_name == "I6" else 0.35
        binary_image = norm_image < threshold
        
        # Apply morphological operations to clean up the image
        binary_image = binary_dilation(binary_image, structure=morphology_disk(2))
        binary_image = binary_fill_holes(binary_image)
        
        # Try to find the humerus region
        # ... (código específico para estas imágenes)
        
    # Find the center and radius of the humerus using Hough Circle Transform
    center_x, center_y, radius = detect_humerus_circle(pixel_array)
    
    # If detection failed but we have a previous contour, use it as reference
    if (center_x is None or radius is None) and previous_contour is not None:
        # Calculate center and average radius from previous contour
        prev_center = np.mean(previous_contour, axis=0)
        prev_radius = np.mean(np.sqrt(np.sum((previous_contour - prev_center)**2, axis=1)))
        
        # Use previous contour with slight adjustments
        init_contour = previous_contour.copy()
    elif center_x is not None and radius is not None:
        # Generate initial circular contour from Hough detection
        init_contour = generate_circular_contour(center_x, center_y, radius)
    else:
        # If all detection methods fail, return empty contour
        return np.array([])
    
    # Prepare image for snake algorithm (inverted gradient)
    # Higher alpha (tension) and beta (rigidity) to prevent collapse
    # These parameters were increased to maintain coherent shape
    alpha = 0.015  # Tension parameter
    beta = 10.0    # Rigidity parameter 
    
    # Apply active contour model (snake)
    try:
        # Use the active contour algorithm
        snake = active_contour(
            gaussian_filter(image_processing.normalize_image(pixel_array), 2.0),
            init_contour,
            alpha=alpha,      # Tension
            beta=beta,        # Rigidity
            gamma=0.001,      # Time step
            w_line=0,         # Weight for image intensity
            w_edge=1.0,       # Weight for image edges
            max_iterations=100
        )
        
        return snake
    except:
        # If snake algorithm fails, return the initial contour
        return init_contour
\end{lstlisting}

\subsection{Ajuste de splines}

El siguiente fragmento muestra la implementación del ajuste mediante B-splines, que fue clave para obtener contornos suaves:

\begin{lstlisting}[language=Python, caption=Implementación del ajuste mediante B-splines en spline/fitting.py]
def apply_bspline(contour: np.ndarray, degree: int = 3, smoothing: float = 10.0) -> Tuple[np.ndarray, np.ndarray]:
    """
    Apply B-spline fitting to a contour.
    
    Args:
        contour: Input contour points as numpy array of shape (n_points, 2)
        degree: Degree of the B-spline
        smoothing: Smoothing factor (0 = interpolation)
        
    Returns:
        Tuple containing arrays of x and y coordinates of the spline
    """
    try:
        # Extract x and y coordinates
        y, x = contour[:, 0], contour[:, 1]
        
        # Arc length parametrization
        t = np.zeros(len(x))
        t[1:] = np.cumsum(np.sqrt(np.diff(x)**2 + np.diff(y)**2))
        t = t / t[-1]
        
        # Check if we have enough points for the spline degree
        if len(x) <= degree:
            # If not enough points, reduce degree
            degree = min(degree, max(1, len(x) - 1))
        
        # Periodic B-spline fitting
        tck, u = splprep([x, y], u=t, s=smoothing, k=degree, per=1)
        
        # Evaluate the spline at uniform parameter values
        u_new = np.linspace(0, 1, 200)
        x_spline, y_spline = splev(u_new, tck)
        
        return x_spline, y_spline
    except Exception as e:
        print(f"Error applying B-spline: {e}")
        return None, None
\end{lstlisting}

\subsection{Generación del modelo 3D}

La generación del modelo 3D a partir de los contornos extraídos:

\begin{lstlisting}[language=Python, caption=Generación del modelo 3D a partir de los contornos en run\_visualization.py]
def generate_3d_model(contours, z_spacing, pixel_spacing, output_path):
    """
    Generate a 3D model from stacked contours.
    
    Args:
        contours: List of contours for each slice
        z_spacing: Spacing in mm between slices
        pixel_spacing: Pixel resolution (x, y) in mm
        output_path: Path to save the 3D model
    """
    # Filter empty contours
    valid_contours = [c for c in contours if len(c) > 0]
    
    if not valid_contours:
        print("No valid contours to generate the 3D model.")
        return
    
    # Create 3D figure
    fig = plt.figure(figsize=(12, 10))
    ax = fig.add_subplot(111, projection='3d')
    
    # For uniform visualization, normalize each contour to an equal number of points
    n_points = 100
    
    # Colormap to color the contours by height
    colors = plt.cm.viridis(np.linspace(0, 1, len(valid_contours)))
    
    # Create 3D surface
    for i, contour in enumerate(valid_contours):
        # Z coordinate of the slice (based on DICOM spacing)
        z = i * z_spacing
        
        # Scale XY coordinates according to pixel resolution
        x = contour[:, 1] * pixel_spacing[0]
        y = contour[:, 0] * pixel_spacing[1]
        
        # Resample the contour with B-splines to have uniform points
        if len(contour) > 3:
            try:
                from scipy.interpolate import splprep, splev
                
                # Arc length parameterization
                t = np.zeros(len(x))
                t[1:] = np.cumsum(np.sqrt(np.diff(x)**2 + np.diff(y)**2))
                t = t / t[-1]
                
                # Closed B-spline
                tck, u = splprep([x, y], u=t, s=0, per=1)
                u_new = np.linspace(0, 1, n_points)
                x_new, y_new = splev(u_new, tck)
                
                # Draw contour
                ax.plot(x_new, y_new, [z]*len(x_new), '-', color=colors[i], alpha=0.7, linewidth=2)
                
                # Connect with previous contour if it exists
                if i > 0 and len(valid_contours[i-1]) > 3:
                    x_prev = valid_contours[i-1][:, 1] * pixel_spacing[0]
                    y_prev = valid_contours[i-1][:, 0] * pixel_spacing[1]
                    z_prev = (i-1) * z_spacing
                    
                    # Resample previous contour
                    t_prev = np.zeros(len(x_prev))
                    t_prev[1:] = np.cumsum(np.sqrt(np.diff(x_prev)**2 + np.diff(y_prev)**2))
                    t_prev = t_prev / t_prev[-1]
                    
                    tck_prev, u_prev = splprep([x_prev, y_prev], u=t_prev, s=0, per=1)
                    x_prev_new, y_prev_new = splev(u_new, tck_prev)
                    
                    # Connect corresponding points between contours
                    for j in range(0, n_points, 5):  # Connect every 5 points to avoid overloading
                        ax.plot([x_prev_new[j], x_new[j]], [y_prev_new[j], y_new[j]], 
                                [z_prev, z], '-', color='gray', alpha=0.3, linewidth=1)
            except Exception as e:
                print(f"Error processing contour {i}: {e}")
                continue
\end{lstlisting} 

\end{document} 
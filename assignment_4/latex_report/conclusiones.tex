\section{Conclusiones}

El desarrollo e implementación del sistema de segmentación y modelado 3D del húmero ha permitido obtener valiosas conclusiones sobre el procesamiento de imágenes médicas y la reconstrucción tridimensional de estructuras anatómicas.

\subsection{Principales logros}

Se han alcanzado los siguientes logros principales:

\begin{enumerate}
    \item \textbf{Segmentación precisa:} Se ha logrado una segmentación precisa del húmero en imágenes de tomografía computarizada, incluso en regiones anatómicamente complejas.
    
    \item \textbf{Coherencia anatómica:} Se ha conseguido mantener la coherencia anatómica entre cortes consecutivos, generando una representación tridimensional anatómicamente correcta.
    
    \item \textbf{Robustez del algoritmo:} El algoritmo desarrollado ha demostrado ser robusto ante variaciones en la calidad de la imagen y cambios en la morfología del húmero.
    
    \item \textbf{Visualización efectiva:} El modelo 3D generado permite una visualización efectiva y comprensible de la estructura anatómica del húmero.
\end{enumerate}

\subsection{Implicaciones técnicas}

Desde el punto de vista técnico, el proyecto ha demostrado:

\begin{enumerate}
    \item La efectividad de combinar técnicas de detección inicial (transformada de Hough) con métodos de refinamiento de contorno (snake) para segmentación médica.
    
    \item La importancia de la transferencia de información entre cortes consecutivos para mantener la coherencia anatómica.
    
    \item El valor de los B-splines como herramienta para obtener representaciones suaves y precisas de estructuras anatómicas.
    
    \item La necesidad de adaptar los algoritmos a las características específicas de diferentes regiones anatómicas.
\end{enumerate}

\subsection{Impacto potencial}

El sistema desarrollado tiene potencial para impactar en diversos campos:

\begin{itemize}
    \item \textbf{Planificación quirúrgica:} La reconstrucción 3D precisa del húmero puede ser útil para la planificación de intervenciones quirúrgicas.
    
    \item \textbf{Educación médica:} Los modelos generados pueden utilizarse como herramientas educativas para estudiantes de medicina y profesionales de la salud.
    
    \item \textbf{Investigación biomédica:} La metodología puede adaptarse para estudios de variabilidad anatómica o análisis de patologías específicas.
    
    \item \textbf{Desarrollo de prótesis:} Los modelos 3D precisos pueden contribuir al diseño de prótesis personalizadas.
\end{itemize}

\subsection{Lecciones aprendidas}

El desarrollo del proyecto ha proporcionado valiosas lecciones:

\begin{enumerate}
    \item \textbf{Importancia del enfoque iterativo:} La identificación de problemas específicos y su resolución mediante mejoras incrementales ha sido clave para el éxito del proyecto.
    
    \item \textbf{Valor del conocimiento anatómico:} La comprensión de la anatomía del húmero ha sido fundamental para evaluar los resultados y guiar las mejoras del algoritmo.
    
    \item \textbf{Necesidad de redundancia:} La implementación de mecanismos de respaldo para cuando la detección falla ha aumentado significativamente la robustez del sistema.
    
    \item \textbf{Equilibrio entre parametrización manual y automatización:} Si bien es deseable un alto grado de automatización, ciertos parámetros requieren ajuste manual para adaptarse a diferentes casos.
\end{enumerate}

\subsection{Consideraciones finales}

El proyecto demuestra que es posible obtener reconstrucciones 3D anatómicamente precisas a partir de imágenes de tomografía computarizada utilizando técnicas de procesamiento de imagen y modelado geométrico. La metodología desarrollada proporciona una base sólida que puede extenderse a otras estructuras anatómicas y aplicaciones médicas.

La combinación de algoritmos de detección de contornos, contornos activos, ajuste mediante splines y visualización 3D ha demostrado ser efectiva para la segmentación y modelado del húmero, superando desafíos como la coherencia anatómica entre cortes y la precisión en regiones complejas.

Este trabajo establece una metodología viable para la reconstrucción tridimensional a partir de imágenes médicas que podría integrarse en sistemas de apoyo a la práctica clínica, la educación médica y la investigación biomédica. 